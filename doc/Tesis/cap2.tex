\chapter{Estado del Arte}
\label{chap:edoArt} %A partir de aqui es un comentario: esta etiqueta sirve para poder referirnos al documento desde otro


\section{Marco te�rico conceptual.}
\subsection{Realidad Aumentada.}

Texo (Ver Figura \ref{fig:milgram}). 
\begin{figure}[htp]
  \centering
  \includegraphics[width=.9\linewidth]{Imagenes/continuoMilgram}
  \caption {Continuo de Milgram.}{\label{fig:milgram} Continuo de Milgram, una forma de clasificar los sistemas de acuerdo al grado de inmersi�n del usuario.}
\end{figure}

As� se pone una lista
\begin{itemize}
\item Elemento1
\item Elemento2
\item ElementoN
\end{itemize}

As� se pone una lista numerada:
\begin{enumerate}
\item Elemento1
\item Elemento2
\item ElementoN
\end{enumerate}

Esto es una tabla~\ref{tab:range}:

\begin{table} [ht]
\renewcommand{\arraystretch}{1.1}
\caption{Rango de seguimiento ARToolKit}
{\label{tab:range}}%Rango de seguimiento de diferentes tama\~nos de patrones de ARToolKit}
\centering
\begin{tabular}{c|c}
    \hline
    Tama\~no del patr\'on (cm.) & Rango de usabilidad (cm.) \\
    \hline
    \hline
		6.98	& 40.64 \\
		%\hline
		8.89	& 63.5 \\
		%\hline
		10.79	& 86.36 \\
		%\hline
		18.72	& 127 \\
		\hline
\end{tabular}
\end{table}

